\begin{thm}
Пусть $R_1\leq r$ п.н. для некоторого $r>0$, $d\geq 1$, $n \gg a^d\lambda$, и $n\gg a^d$. 
Предположим, что 
\begin{equation*}
    \PP[K_1 \geq n] \geq \exp \left(-\beta\cdot n\log n + \gamma\cdot n\log\lambda + O(n)\right),\  n \to \infty.
 \end{equation*}{}
Тогда 
\begin{equation*}
    \PP[K_a \geq n] \geq \exp \left(-\beta\cdot n\log n + \beta\cdot dn\log a + \gamma\cdot n\log\lambda + O(n)\right),\  n, a \to \infty.
\end{equation*}{}
\end{thm}

\begin{proof}
Пусть $h = h(a, r) = \left\lfloor\dfrac{a-1}{2r+1}\right\rfloor$. Рассмотрим набор ячеек:
\begin{equation*}
    \left\{\prod_{m = 1}^d\left[k_m\left(\dfrac{a - 2rh}{h+1} + 2r\right), k_m\left(\dfrac{a - 2rh}{h+1} + 2r\right) + \dfrac{a - 2rh}{h+1}\right]\right\},
\end{equation*}{}
где $k_m\in \{0, 1, \ldots, h\}$.

Заметим, что для точек $x$ и $y$, лежащих в разных ячейках, $\min_{1\leq m\leq d} |x^{(m)} - y^{(m)}| \geq 2r$, поэтому шары с центрами в разных ячейках не пересекаются. 

Мы построили ячейки $V_1, \ldots, V_l$, где $l = (h+1)^d$. Назовём минимальным числом видимых шаров в ячейке $V_j$ следующую величину:
\begin{equation*}
    K^j = \min\{s\ |\ \exists i_1, \ldots i_s \in \{1, \ldots N\} \colon S \cap V_j = \bigcup_{t = 1}^s B(\xi_{i_t}, R_1)\cap V_j \}.
\end{equation*}{}
Обозначим $V = \bigcup_{j=1}^l V_j$. 
Определим следующее событие:
\begin{equation*}
    E = \{\xi_i \in V \text{ для } i = 1, \ldots, N\} \cap \bigcap_{j = 1}^l\{K^j \geq n/(h+1)^d\}.
\end{equation*}{}
Заметим, что для различных $j$ события $\{K^j \geq n/(h+1)^d\ |\ \xi_i \in V \text{ для } i = 1, \ldots, N\}$ независимы.

Докажем, что событие $E$ влечет событие $\{K_a \geq n\}$. Действительно, если все центры лежат в наших ячейках, то шары с центрами в разных ячейках не пересекаются. Тогда если в каждой ячейке минимальное число видимых шаров хотя бы $n/(h+1)^d$, то минимальное число видимых шаров во всей картинке хотя бы $n$.

Так как количество центров шаров в $[0, a]^d \setminus V$ -- пуассоновская случайная величина с параметром $\lambda\vol_d([0, a]^d\setminus V)$, можно выписать следующую оценку:
\begin{equation*}
    \PP\{\xi_i \in V \text{ для } i = 1, \ldots, N\} = \exp(-\lambda\cdot\vol_d([0,a]^d\setminus V)) \geq \exp(-\lambda\cdot (2ra^{d-1}) \cdot dh) \geq \exp(-\lambda\cdot da^d).
\end{equation*}{}

Заметим, что при условии, что центры шаров лежат только в выбранных ячейках, события $\{K^j\geq n/(h+1)^d\}$ -- это с точностью до гомотетии то же самое, что событие $\{\til K_1 \geq n/(h+1)^d\}$, где $\til K_1$ -- минимальное необходимое количество шаров для задачи в единичном кубе с постоянными радиусами $\til R_1 = R_1\cdot \left(\dfrac{a-2rh}{h+1}\right)^{-1}$ и интенсивностью $\til\lambda = \lambda\cdot\left(\dfrac{a-2rh}{h+1}\right)^d$. Таким образом,  
\begin{multline*}
    \PP\{K^j \geq n/(h+1)^d\ | \ \xi_i \in V \text{ для } i = 1, \ldots, N\} =
    \PP\{\til K_1 \geq n/(h+1)^d\} \geq\\ 
    \geq \exp \left(-\beta\cdot n/(h+1)^d\log (n/(h+1)^d) + \gamma\cdot n/(h+1)^d\log\til\lambda + O(n/(h+1)^d)\right) \geq\\
    \geq \exp \left(-\beta\cdot n/(h+1)^d(\log n - d\log a) + \gamma\cdot n/(h+1)^d\log\lambda + O(n/(h+1)^d)\right).
\end{multline*}{}

Здесь мы использовали, что $n\log\til\lambda = n\log\lambda + O(n)$. Также, используя то, что $(h~+~1)^d\cdot O(n/(h~+~1)^d) = O(n)$,  получаем 
\begin{multline*}
    \PP\{K_a \geq n\} \geq \exp \left(-\lambda da^d-\beta\cdot n(\log n - d\log a) + \gamma\cdot n\log\lambda + O(n)\right) = \\
    \exp \left(-\beta\cdot n\log n + \beta\cdot dn\log a + \gamma\cdot n\log\lambda + O(n)\right).
\end{multline*}{}
\end{proof}{}

\begin{cor}[теорема \ref{lowconst1}]
Пусть $R_1\equiv r > 0$ п.н., где шары берутся в $\ell_1$-норме, $d\geq 2$, $n \gg a^d\lambda$, и $n\gg a^d$. 
Тогда $$\PP[K_a \geq n] \geq \exp \left(-\left(1+\dfrac{1}{d-1}\right)n\log n + \left(1+\dfrac{1}{d-1}\right)dn\log a  +n\log\lambda + O(n)\right),\  n, a\to \infty.$$
\end{cor}{}

\begin{proof}
    Эта оценка следует из proposition 10 в \cite{AL}. Из доказательства этого утверждения видно, что есть оценка 
    $$\PP[K_1 \geq n] \geq \exp \left(-\left(1+\dfrac{1}{d-1}\right)n\log n   +n\log\lambda + O(n)\right),\  n \to \infty.$$ Применяя доказанную выше теорему, получаем требуемое.
\end{proof}{}

\begin{cor}[теорема \ref{lowconst2}]
Пусть $R_1\equiv r > 0$ п.н., где шары берутся в $\ell_2$-норме, $d\geq 2$, $n \gg a^d\lambda$, и $n\gg a^d$. 
Тогда $$\PP[K_a \geq n] \geq \exp \left(-\left(1+\dfrac{2}{d-1}\right)n\log n + \left(1+\dfrac{2}{d-1}\right)dn\log a  +n\log\lambda + O(n)\right),\  n, a\to \infty.$$
\end{cor}{}

\begin{proof}
    Эта оценка следует из proposition 9 в \cite{AL}. Из доказательства этого утверждения видно, что есть оценка 
    $$\PP[K_1 \geq n] \geq \exp \left(-\left(1+\dfrac{2}{d-1}\right)n\log n   +n\log\lambda + O(n)\right),\  n \to \infty.$$ Применяя доказанную выше теорему, получаем требуемое.
\end{proof}{}

\begin{cor}
Пусть $R_1\equiv r > 0$ п.н., где шары берутся в $\ell_\infty$-норме, $d\geq 2$, $n \gg a^d\lambda$, и $n\gg a^d$. 
Тогда $$\PP[K_a \geq n] \geq \exp \left(-\left(1+\dfrac{1}{d-1}\right)n\log n + \left(1+\dfrac{1}{d-1}\right)dn\log a  +n\log\lambda + O(n)\right),\  n, a\to \infty.$$
\end{cor}{}

\begin{proof}
    Эта оценка следует из proposition 11 в \cite{AL}. Из доказательства этого утверждения видно, что есть оценка 
    $$\PP[K_1 \geq n] \geq \exp \left(-\left(1+\dfrac{1}{d-1}\right)n\log n   +n\log\lambda + O(n)\right),\  n \to \infty.$$ применяя доказанную выше теорему, получаем требуемое.
\end{proof}{}


