\documentclass[12pt]{article}
\usepackage{fullpage}   
\usepackage[utf8]{inputenc}
\usepackage[russian]{babel}
\usepackage{amsthm, amsmath, amssymb, pdfpages}

%\usepackage[left=3cm,right=1cm,
%   top=2cm,bottom=2cm,bindingoffset=0cm]{geometry}



\renewcommand{\proofname}{Доказательство}
\theoremstyle{plain}
\newtheorem{thm}{Теорема} %[section], чтобы нумеровать сначала в каждом разделе
\newtheorem{lm}{Лемма}
\newtheorem*{st}{Утверждение}

\theoremstyle{definition}
\newtheorem*{defn}{Определение}
\newtheorem*{ex}{Упр}
\newtheorem*{cor}{Следствие}
\newtheorem*{name}{Обозначение}

\theoremstyle{remark}
\newtheorem*{rem}{Замечание}


\def\geq{\geqslant}
\def\ge{\geqslant}
\def\leq{\leqslant}
\def\le{\leqslant}

\DeclareMathOperator{\supp}{supp}
\DeclareMathOperator{\Id}{Id}
\DeclareMathOperator{\D}{D}
\DeclareMathOperator{\vol}{vol}

\newcommand{\cuplim}{\bigcup\limits}
\newcommand{\ilim}{\int\limits}
\newcommand{\slim}{\sum\limits}
\newcommand{\maxlim}{\max\limits}
\newcommand{\suplim}{\sup\limits}
\newcommand{\T}{\mathbb{T}}
\newcommand{\dm}{\, d\mu_n}
\newcommand{\R}{\mathbb{R}}
\newcommand{\PP}{\mathbb{P}}
\newcommand{\Z}{\mathbb{Z}}
%\newcommand{\R^nn}{\mathbb{R}^n}
%\newcommand{C^1}{\mathbb{C}^1}
\newcommand{\til}{\widetilde}
\newcommand{\dd}{\partial}
\newcommand{\eps}{\varepsilon}



\begin{document}

\begin{thm}
Пусть $R_1\equiv r > 0$ п.н., где шары берутся в $\ell_1$-норме, $d\geq 2$ и $n \gg a^d\lambda$. 
Предположим, что 
\begin{equation*}
    \PP[K_1 \geq n] \geq \exp \left(-\beta\cdot n\log n + n\log\lambda + O(n)\right),\  n \to \infty.
\end{equation*}{}
Тогда 
\begin{equation*}
    \PP[K_a \geq n] \geq \exp \left(-\beta\cdot n\log n + \beta\cdot dn\log a + n\log\lambda + O(n)\right),\  n, a \to \infty.
\end{equation*}{}
\end{thm}

\begin{proof}
Пусть $h = h(a, r) = \left\lfloor\dfrac{a-1}{2r+1}\right\rfloor$. Рассмотрим набор ячеек:
\begin{equation*}
    \left\{\prod_{m = 1}^d\left[k_m\left(\dfrac{a - 2rh}{h+1} + 2r\right), k_m\left(\dfrac{a - 2rh}{h+1} + 2r\right) + \dfrac{a - 2rh}{h+1}\right]\right\},
\end{equation*}{}
где $k_m\in \{0, 1, \ldots, h\}$.

Заметим, что для точек $x$ и $y$, лежащих в разных ячейках, $\min_{1\leq m\leq d} |x^{(m)} - y^{(m)}| \geq 2r$, поэтому шары с центрами в разных ячейках не пересекаются. 

Также заметим, что длина стороны ячейки удовлетворяет следующей оценке:
\begin{equation*}
    \dfrac{a - 2rh}{h+1} \geq \dfrac{a - 2r\cdot \frac{a-1}{2r+1}}{\frac{a-1}{2r+1} + 1} = 1.
\end{equation*}{}

Мы построили ячейки $V_1, \ldots, V_l$, где $l = (h+1)^d$. Назовём минимальным числом видимых шаров в ячейке $V_j$ следующую величину:
\begin{equation*}
    K^j = \min\{s\ |\ \exists i_1, \ldots i_s \in \{1, \ldots N\} \colon S \cap V_j = \bigcup_{t = 1}^s B(\xi_{i_t}, R_1)\cap V_j \}.
\end{equation*}{}
Введем еще одно обозначение: $U = [0,a]^d \setminus (\bigcup_{j=1}^l V_j)$.

Определим следующее событие:
\begin{equation*}
    E = \{\xi_i \not\in U \text{ для } i = 1, \ldots, N\} \cap \bigcup_{j = 1}^l\{K^j \geq n/(h+1)^d\}.
\end{equation*}{}
Заметим, что для различных $j$ события $\{K^j \geq n/(h+1)^d\}$ независимы.

Докажем, что событие $E$ влечет событие $\{K_a \geq n\}$. Действительно, если все центры лежат в наших ячейках, то шары с центрами в разных ячейках не пересекаются. Тогда если в каждой ячейке минимальное число видимых шаров хотя бы $n/(h+1)^d$, то минимальное число видимых шаров во всей картинке удовлетворяет неравенству:
\begin{equation*}
    K_a \geq l\cdot n/(h+1)^d = (h+1)^d \cdot n/(h+1)^d  = n.
\end{equation*}{}

Вычислим вероятности искомых событий:
\begin{equation*}
    \PP\{\xi_i \not\in U \text{ для } i = 1, \ldots, N\} =\exp(-\lambda\cdot\vol_d(U)) \geq \exp(-\lambda\cdot (2ra^{d-1}) \cdot dh) \geq \exp(-\lambda\cdot da^d).
\end{equation*}{}

Заметим, что так как сторона ячеек хотя бы 1, то выполнено неравенство:
\begin{multline*}
    \PP\{K^j \geq n/(h+1)^d\} \geq 
    \PP\{K_1 \geq n/(h+1)^d\} \geq\\ 
    \geq \exp \left(-\beta\cdot n/(h+1)^d\log (n/(h+1)^d) + n/(h+1)^d\log\lambda + O(n/(h+1)^d)\right) \geq\\
    \geq \exp \left(-\beta\cdot n/(h+1)^d(\log n - d\log a) + n/(h+1)^d\log\lambda + O(n/(h+1)^d)\right).
\end{multline*}{}

Собирая все вместе, получаем 
\begin{multline*}
    \PP\{K_a \geq n\} \geq \exp \left(-d\lambda a^d-\beta\cdot n(\log n - d\log a) + n\log\lambda + O(n)\right) = \\
    \exp \left(-\beta\cdot n\log n - \beta\cdot dn\log a + n\log\lambda + O(n)\right).
\end{multline*}{}
\end{proof}{}




\end{document}
