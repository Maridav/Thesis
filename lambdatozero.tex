\section{Поведение $K_a$ при малых $\lambda$}

В этом разделе речь пойдет о случае, когда интенсивность пуассоновского точечного процесса $\lambda$ стремится к нулю при $a\to\infty$.

\begin{thm}
Пусть $\lambda = \lambda(a)$ таково, что $a^d\lambda^2 \to 0$ при $a\to\infty$, и случайная величина $R_1$ имеет $d$-й момент, $d\geq 1$. Тогда $\PP[K_a < N] \to 0$ при $a\to\infty$.
\end{thm}

\begin{proof}
Рассмотрим событие $$E =\bigcup_{i\in \{1, \ldots N\}} \bigcup_{\substack{j\in\{1, \ldots, N\}\colon\\ j \not= i}}\{\xi_i\in B(\xi_j, R_j)\}.$$
Так как центры шаров распределены равномерно в $[0, a]^d$ и величины $R_i$ независимы и одинаково распределены, выполнено равенство:
\begin{equation*}
    \PP[\xi_i\in B(\xi_j, R_j)] =
    \dfrac{1}{a^d}\E \vol_d [B(\xi_j, R_j) \cap [0,a]^d] \leq
    \dfrac{1}{a^d}\E \vol_d B(\xi_j, R_j) = \dfrac{1}{a^d}\E \vol_d B(0, R_1) = \dfrac{c}{a^d}
\end{equation*}
в силу того, что у $R_1$ существует $d$-й момент.

Заметим, что если $K_a < N$, то событие $E$ выполняется, так как есть шар, лежащий в объединении других шаров. Поэтому 
\begin{multline*}
    \PP[K_a < N]\leq
    \PP[E] =
    \sum_{k=0}^\infty \PP[E|N = k]\PP[N = k] \leq\\
    \leq \sum_{k=0}^\infty \left(\sum_{1 \leq i \leq k}\sum_{\substack{1\leq j\leq k\colon\\ j\not= i}} \PP[\xi_i\in B(\xi_j, R_j)] \right)\PP[N = k] \leq 
    \sum_{k=0}^\infty k(k-1) \cdot\dfrac{c}{a^d}\cdot\dfrac{(a^d\lambda)^k}{k!}e^{-a^d\lambda} = \\
    = \dfrac{c}{a^d}(a^d\lambda)^2 = 
    ca^d\lambda^2 \to 0, \ a\to\infty.
\end{multline*}
\end{proof}

\begin{cor}
Если последовательность $a^d\lambda$ ограничена и случайная величина $R_1$ имеет $d$-й момент, то $\PP[K_a < N] \to 0, \ a \to\infty$.
\end{cor}

Таким образом, в данном случае величина $K_a$ асимптотически ведет себя как $N$.
