
Следующая теорема верна для любой нормы. 
\begin{thm}\label{uppgen}
Пусть $d\geq 1$,  $n \gg a^d\lambda$ и $n \gg a^d$. 
Предположим, что неравенство 
\begin{equation*}
    \PP[K_1 \geq n] \leq \exp \left(-\beta\cdot n\log n + \gamma\cdot n\log\lambda + O(n)\right),\  n \to \infty
\end{equation*}{}
верно для радиусов $R_1$ с некоторым распределением и для радиусов $\til R_1 = R_1\cdot b$ одновременно для любых $b\in (0, 1]$.
Тогда 
\begin{equation*}
    \PP[K_a \geq n] \leq \exp \left(-\beta\cdot n\log n + \beta\cdot dn\log a + \gamma\cdot n\log\lambda + O(n)\right),\  n, a \to \infty.
\end{equation*}{}
\end{thm}

\begin{proof}
Разобьем куб $[0, a]^d$ на такие ячейки:
\begin{equation*}
    \prod_{m = 1}^d\left[\dfrac{k_m a}{\lfloor a \rfloor}, \dfrac{(k_m + 1)a}{\lfloor a \rfloor}\right],
\end{equation*}{}
где $k_m\in \{0, 1, \ldots, \lfloor a\rfloor - 1\}$. Пусть это ячейки $V_1, \ldots, V_l$, где $l = \lfloor a\rfloor ^d$. 
Определим картинку, образованную ячейкой $V_j$, следующим образом:
\begin{equation*}
    S^j = \cuplim_{\substack{i\in \{1, \ldots, N\}\colon\\ \xi_i\in V_j}} B(\xi_i, R_i) \cap V_j.
\end{equation*}{}
Назовём минимальным числом видимых шаров в ячейке $V_j$ следующую величину:
\begin{equation*}
    K^j = \min\{s\geq 1\ |\ \exists i_1, \ldots i_s \in \{1, \ldots, N\} \colon \xi_{i_t}\in V_j,\ t = 1, \ldots, s;\ S^j = \bigcup_{t = 1}^s B(\xi_{i_t}, R_1)\cap V_j\}.
\end{equation*}{}
Рассмотрим событие
\begin{equation*}
    E = \bigcup_{\substack{\{n_1, \ldots n_l\}\colon\\ n_s \geq 0\ \forall s,\\ \sum n_s = n}} \bigcap_{s = 1}^l \{K^s \geq n_s\}.
\end{equation*}{}
Заметим, что из события $\{K_a \geq n\}$ следует событие $E$. Действительно, если $K_a \geq n$, то и $\sum K^s \geq n$. Тогда существует набор $\{n_1, \ldots n_l\}$ с $n_s \geq 0\ \forall s$ и $\sum n_s = n$, для которого выполнено $K^s \geq n_s$ для всех $s = 1, \ldots l$. Это и есть событие $E$.

Вычислим вероятность события $\{K^s \geq n_s\}$. 
С точностью до гомотетии это то же самое, что $\til K_1 \geq n_s$, где $\til K_1$ обозначает минимальное число видимых шаров в кубе $[0, 1]^d$, рассматриваемое раньше, но с интенсивностью пуассоновского поля $\til\lambda = \lambda (a/\lfloor a\rfloor)^d$ и радиусами $\til R_1 = R_1(\lfloor a\rfloor/a)$. 

Заметим также, что события $\{K^s \geq n_s\}$ независимы для различных $s$.\\
Таким образом,
\begin{multline*}
    \PP[K_a\geq n] \leq
    \PP[E] \leq 
    \sum_{\substack{\{n_1, \ldots n_l\}\colon\\ n_s \geq 0\ \forall s,\\ \sum n_s = n}}\prod_{s=1}^l\PP[K^s\geq n_s] =
    \sum_{\substack{\{n_1, \ldots n_l\}\colon\\ n_s \geq 0\ \forall s,\\ \sum n_s = n}}\prod_{s=1}^l\PP[\til K_1\geq n_s]\leq \\
    \leq \sum_{\substack{\{n_1, \ldots n_l\}\colon\\ n_s \geq 0\ \forall s,\\ \sum n_s = n}}\prod_{s=1}^l \exp(-\beta\cdot n_s\log n_s + \gamma\cdot n_s\log\til\lambda + O(n_s) =\\
    =\sum_{\substack{\{n_1, \ldots n_l\}\colon\\ n_s \geq 0\ \forall s,\\ \sum n_s = n}}\exp\left(-\beta\cdot \sum_{s=1}^l n_s\log n_s + \gamma\cdot n\log\til\lambda + \sum_{s=1}^l O(n_s)\right).
    \end{multline*}
Нетрудно убедиться, что минимум выражения $\sum_{s=1}^l n_s\log n_s$ достигается на наборе $\{n/l, \ldots, n/l\}$. Также заметим, что $n\log\til\lambda = n\log\lambda + o(n)$ и $\sum_{s=1}^lO(n_s) = O(n))$. Получим, что
    \begin{multline*}
    \PP[K_a\geq n]\leq
    \sum_{\substack{\{n_1, \ldots n_l\}\colon\\ n_s \geq 0\ \forall s,\\ \sum n_s = n}}\exp\left(-\beta\cdot \sum_{s=1}^l \dfrac{n}{l}\log\dfrac{n}{l} + \gamma\cdot n\log\lambda + O(n)\right) = \\
    =\sum_{\substack{\{n_1, \ldots n_l\}\colon\\ n_s \geq 0\ \forall s,\\ \sum n_s = n}}\exp\left(-\beta\cdot n\log n + \beta\cdot dn\log a + \gamma\cdot n\log\lambda + O(n)\right).
    \end{multline*}
Посчитаем количество способов разбиения числа $n$ на $\lfloor a\rfloor^d$ неотрицательных слагаемых:
\begin{equation*}
    \binom{n + \lfloor a\rfloor^d - 1}{\lfloor a\rfloor^d - 1} \leq \dfrac{(n + a^d)^{a^d}}{(\lfloor a\rfloor^d -1)!} = \exp\left(a^d + a^d\log\left(\dfrac{n}{a^d}+1\right) - \dfrac{d}{2} \log a + o(n)\right) = \exp(o(n)).
\end{equation*}{}
Объединяя все полученные неравенства, получаем, что 
\begin{equation*}
    \PP[K_a\geq n]\leq
    \exp\left(-\beta\cdot n\log n + \beta\cdot dn\log a + \gamma\cdot n\log\lambda + O(n) \right).
\end{equation*}{}
\end{proof}{}

\begin{cor}[теорема \ref{uppconst1}]
Пусть $R_1\equiv r > 0$ п.н., где шары берутся в $\ell_1$-норме, $d\geq 2$, $n \gg a^d\lambda$, и $n\gg a^d$. 
Тогда $$\PP[K_a \geq n] \leq \exp \left(-\left(1+\dfrac{1}{d-1}\right)n\log n + \left(1+\dfrac{1}{d-1}\right)dn\log a  +n\log\lambda + O(n)\right),\  n, a\to \infty.$$
\end{cor}{}

\begin{proof}
    Эта оценка следует из proposition 19 в \cite{AL}. Из доказательства этого утверждения видно, что есть оценка 
    $$\PP [K_1 \geq n] \leq \exp \left(-\left(1+\dfrac{1}{d-1}\right)n\log n   +n\log\lambda + O(n)\right),\  n \to \infty.$$ Применяя доказанную выше теорему, получаем требуемое.
\end{proof}{}

\begin{cor}
Пусть $R_1\equiv r > 0$ п.н., где шары берутся в $\ell_2$-норме, $d\geq 2$, $n \gg a^d\lambda$, и $n\gg a^d$. 
Тогда $$\PP[K_a \geq n] \leq \exp \left(-\left(1+\dfrac{1}{d-1}\right)n\log n + \left(1+\dfrac{1}{d-1}\right)dn\log a  +n\log\lambda + O(n)\right),\  n, a\to \infty.$$
\end{cor}{}

\begin{proof}
    Эта оценка следует из proposition 21 в \cite{AL}. Из доказательства этого утверждения видно, что есть оценка 
    $$\PP[K_1 \geq n] \leq \exp \left(-\left(1+\dfrac{1}{d-1}\right)n\log n   +n\log\lambda + O(n)\right),\  n \to \infty.$$ Применяя доказанную выше теорему, получаем требуемое.
\end{proof}{}

\begin{cor}
Пусть распределение $R_1$ таково, что $\sup_{x>0}\PP[R_1\in [x, x+r]]\leq cr^\alpha$ для некоторых $c>0$, $\alpha\in(0, 1]$ и всех $r>0$. Предположим, что $d \geq 1$, $n\gg a^d$ и $n \gg a^d\lambda$.
Тогда $$\PP[K_a \geq n] \geq \exp\left(-\left(1 + \alpha/d\right)n\log n + (1 + \alpha/d)dn\log a + n\log \lambda + O(n)\right), \ n, a \to \infty.$$
\end{cor}{}

\begin{proof}
    Написать доказательство!!! Для этого надо вытащить лямбду из предложения 18 из статьи.
\end{proof}
